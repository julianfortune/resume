% Adapted from: "Scismic's Recommended CV Template for Biotech and Pharma Jobs" by Scismic and Overleaf, licensed under [CC BY 4.0](https://creativecommons.org/licenses/by/4.0/), accessed April 10, 2020 from [Overleaf](https://www.overleaf.com/latex/templates/scismics-recommended-cv-template-for-biotech-and-pharma-jobs/hbnkjrjnnpjz)

\documentclass[12pt]{article} % decrease fontsize to `10pt` or `11pt` to make more room


%% Dependencies %%

\usepackage{enumitem}
\usepackage{geometry}
\usepackage{microtype}
\usepackage{hyperref}
\usepackage{paracol}
\usepackage{parskip}
\usepackage{soul}  % \ul
\usepackage{textcase}
\usepackage[compact,small,explicit]{titlesec}  % Used for `titleformat`
\usepackage[x11names]{xcolor}

% Depended upon by `pandoc`'s markdown conversion
\providecommand{\tightlist}{%
  \setlength{\itemsep}{0pt}\setlength{\parskip}{0pt}}


%% Fonts %%

\usepackage[defaultsans]{cantarell}
% Changes the default font family to sans-serif (instead of serif)
\renewcommand*\familydefault{\sfdefault}
% Force using an 8-bit encoding (instead of 7)
\usepackage[T1]{fontenc}


%% Formatting Configuration %%

\setlist{leftmargin=*,nosep}

% Make `\ul` from `soul` look like standard underline
\setul{1pt}{.4pt}

% Turn off ugly blue boxes for links
\hypersetup{colorlinks=true, urlcolor=black}


%% Format Headers %%

% Set custom formatting for H1
\titleformat{\section}%
  {\Large}% Font size
  {\thesection}%
  {0pt}%
  {{%
    \fontfamily{Nunito-TLF}\selectfont% Use rounded font
    \textls[15]{% Increase character spacing
      \MakeTextUppercase{#1}%
    }}%
  }%
  [{\color{gray}\titlerule[0.5pt]}] % Horizontal rule beneath text

% Increase H1 margin-top
\titlespacing{\section}{0pt}{10pt}{5pt}  % left / above / below

% Use rounded font for H1
\titleformat{\subsection}%
  {\bfseries\fontfamily{Nunito-TLF}\selectfont}%
  {\thesection}{0pt}{#1}

% Increase H1 margin-top
\titlespacing{\subsection}{0pt}{14pt}{5pt}  % left / above / below

% Make H3's gray & use rounded font
\titleformat{\subsubsection}
   {\small\bfseries\fontfamily{Nunito-TLF}\selectfont}%
   {\thesection}{0pt}%
   {\textcolor{gray}{#1}}


%% Commands %%

% Store a reference to the original `href` command
% Source: https://tex.stackexchange.com/a/47353
\let\defaultref\href

% Override `href` to add an underline via `\ul`, which fixes underlines without
% needing to use `\smash`--which breaks padding
\renewcommand{\href}[2]{%
  \defaultref{#1}{\ul{#2}}%
}

\newcommand{\link}[2]{\href{#1}{#2}}

% Override `href` to add an underline via `\ul`, which fixes underlines without
% needing to use `\smash`--which breaks padding
\renewcommand{\emph}[1]{%
  \textcolor{gray}{#1}%
}

% Byline with contact info / links
\def\separator{ · {}}

\newcommand{\name}[1]{%
  \begin{center}
      {\huge\bfseries\MakeTextUppercase{#1}\par}
  \end{center}
}

%% Document Configuration %%

\geometry{a4paper,margin=1in}
\pagestyle{empty}
\setcounter{secnumdepth}{0}


%% Document metadata %%
% NOTE: Variables come from markdown front-matter via `pandoc` conversion.

\title{Julian Fortune's Resumé}
\author{Julian Fortune}


%% Document definition %%

\begin{document}

% Name (title)
\begin{bfseries}\begin{huge}
  {\fontfamily{Nunito-TLF}\selectfont%
    \MakeTextUppercase{Julian Fortune}
  }
\end{huge}\end{bfseries}

% Contact info
  \link{https://julianfortune.com}{Contact}\separator%
% Public info
    \link{http://github.com/julianfortune}{GitHub}\separator%
\link{http://linkedin.com/in/julianfortune}{LinkedIn}%

% \begin{raggedright}

\vspace{0.5em}

Senior software engineer with 4 years of professional experience,
expertise in leveraging functional techniques and type systems to
prevent bugs, and a track record of success in environments ranging from
startups to massive enterprises.

\hypertarget{skills}{%
\section{Skills}\label{skills}}

\begin{itemize}
\item
  \textbf{Languages:} Python, Typescript, Scala (with \texttt{cats}),
  Kotlin, Swift, \& Haskell.
\item
  \textbf{Cloud:} Object storage, Postgres, MongoDB, instances, \&
  message queues.
\item
  \textbf{Tools:} Docker, Terraform, Debian, \& Make.
\end{itemize}

\hypertarget{experience}{%
\section{Experience}\label{experience}}

\hypertarget{elemint-senior-full-stack-engineer-august-2023present}{%
\subsection{\texorpdfstring{Elemint \emph{· \small Senior Full-stack
Engineer \hfill August
2023--Present}}{Elemint · Senior Full-stack Engineer August 2023--Present}}\label{elemint-senior-full-stack-engineer-august-2023present}}

Hired to help out across the stack and bring functional programming
experience to a customer loyalty application for a top national soccer
league to compliment matches.

\begin{itemize}
\tightlist
\item
  Transformed requirements into concrete technical specifications and
  tickets (e.g., auction system), and collaborated with the design and
  product teams.
\item
  Created Postgres migrations, implemented endpoints, and developed
  components using Radix UI and Tailwind CSS.
\item
  Mentored peers and championed usage of algebraic data types and
  generic typing.
\item
  \textbf{Stack}: React and Typescript frontend; Typescript, NextJS, and
  tRPC backend.
\end{itemize}

\hypertarget{disney-streaming-software-engineer-june-2022august-2023}{%
\subsection{\texorpdfstring{Disney Streaming \emph{· \small Software
Engineer \hfill June 2022--August
2023}}{Disney Streaming · Software Engineer June 2022--August 2023}}\label{disney-streaming-software-engineer-june-2022august-2023}}

Recruited to contribute to a set of internal tools for AWS Kinesis,
including schema registry, code generation, and producer and consumer
SDKs.

\begin{itemize}
\tightlist
\item
  Added Python support to the code generation plugin, which outputs a
  custom library with classes corresponding each event defined in a
  given schema registry.
\item
  Analyzed and optimized AWS usage, resulting in annual savings over
  \$1M.
\item
  Added features to SDK's, `higher-level' services (e.g., snapshotter),
  and SBT plugins.
\item
  Created dashboards that enabled faster resolutions to production
  incidents.
\item
  Wrote documentation and created a Slackbot which reduced support
  workload.
\end{itemize}

\hypertarget{agot-ai}{%
\subsection{Agot AI}\label{agot-ai}}

\vspace{-5pt}

\hypertarget{team-lead-jan-2022-june-2022}{%
\subsubsection{\texorpdfstring{\small Team Lead \hfill Jan 2022 -- June
2022}{Team Lead Jan 2022 -- June 2022}}\label{team-lead-jan-2022-june-2022}}

Promoted to lead the team responsible for processing videos and creating
the labels used for training.

\begin{itemize}
\tightlist
\item
  Planned sprints, conducted interviews, and tracked the team's
  progress.
\item
  Doubled the rate of training data produced through process
  improvements.
\item
  Advocated for strong typing, pure functions, and immutability.
\end{itemize}

\hypertarget{software-engineer-june-2021-jan-2022}{%
\subsubsection{\texorpdfstring{\small Software Engineer \hfill June 2021
-- Jan
2022}{Software Engineer June 2021 -- Jan 2022}}\label{software-engineer-june-2021-jan-2022}}

Hired to tackle integration and algorithms problems, which involved
first reverse-engineering a restaurant display system composed of
android app and microservices, and then implementing several
microservices in order to integrate a computer vision feedback system.

\begin{itemize}
\tightlist
\item
  Wrote Python libraries for standardizing consuming and producing
  RabbitMQ events, and for identifying the closest match for an order
  from a set of candidates.
\item
  Started a Python lecture series to promote best practices.
\end{itemize}

\hypertarget{skyworks-solutions-software-engineer-jan-2021-june-2021}{%
\subsection{\texorpdfstring{Skyworks Solutions \emph{· \small Software
Engineer \hfill Jan 2021 -- June
2021}}{Skyworks Solutions · Software Engineer Jan 2021 -- June 2021}}\label{skyworks-solutions-software-engineer-jan-2021-june-2021}}

Brought on to assist the audio deep learning team by performing machine
learning experiments, administering cloud resources (e.g., MongoDB and
TPU instances), and writing tests.

\begin{itemize}
\tightlist
\item
  Created a Python library for defining weighted finite-state
  transducers and applying operations (e.g., composition) based on
  technical papers.
\end{itemize}

\hypertarget{lucid-software-intern-june-2020-sep-2020}{%
\subsection{\texorpdfstring{Lucid Software \emph{· \small Intern
\hfill June 2020 -- Sep
2020}}{Lucid Software · Intern June 2020 -- Sep 2020}}\label{lucid-software-intern-june-2020-sep-2020}}

Developed new front-end components, redesigned the paywall system to
support a suite of products, and refactored endpoints and models to
support front-end changes.

\begin{itemize}
\tightlist
\item
  Collaborated on a hand-drawn shape detection project that won the
  hackathon.
\item
  \textbf{Stack}: Angular and Typescript frontend, Scala backend.
\end{itemize}

\hypertarget{cbt-nuggets-software-engineer-june-2018-june-2019}{%
\subsection{\texorpdfstring{CBT Nuggets \emph{· \small Software Engineer
\hfill June 2018 -- June
2019}}{CBT Nuggets · Software Engineer June 2018 -- June 2019}}\label{cbt-nuggets-software-engineer-june-2018-june-2019}}

Joined the mobile apps team (part-time during college) to assist with
fixing crashes, debugging memory leaks, implementing features, and
writing tests for the iOS and tvOS apps.

\begin{paracol}{2}
\setlength{\columnsep}{1em}

\hypertarget{education}{%
\section{Education}\label{education}}

\hypertarget{oregon-state-university-20172021}{%
\subsection{Oregon State University ·
2017--2021}\label{oregon-state-university-20172021}}

\vspace{-5pt}

\hypertarget{b.s.-computer-science-4.0-gpa}{%
\subsubsection{B.S., Computer Science (4.0
GPA)}\label{b.s.-computer-science-4.0-gpa}}

Relevant courses: Programming Language Fundamentals (Haskell) \& Deep
Learning

\switchcolumn

\hypertarget{projects}{%
\section{Projects}\label{projects}}

\textbf{\href{https://github.com/julianfortune/CS381Project}{Functional-C}}
--- Statically-typed, interpreted language written in Haskell.

\textbf{\href{https://github.com/Tereshchenkolab/paper-ecg}{PaperECG}}
--- An application that digitizes electrocardiograms (ECGs) built with
PyQt.

\end{paracol}

\hypertarget{publications}{%
\section{Publications}\label{publications}}

\vspace{1pt}

\href{https://doi.org/10.1177/1071181320641076}{Real-Time Speech
Workload Estimation for Intelligent Human-Machine Systems}\\
\emph{Human Factors and Ergonomics Society Annual Meeting, 2020.
Co-authored with Dr.~Jamison Heard and Dr.~Julie A. Adams.}

\href{https://doi.org/10.1101/2021.07.13.21260461}{Digitizing ECG image:
A new method and open-source software code}\\
\emph{Computer Methods and Programs in Biomedicine, June 2022.
Co-authored with Larisa G. Tereshchenko and others.}

% \end{raggedright}

\end{document}
