% Adapted from: "Scismic's Recommended CV Template for Biotech and Pharma Jobs" by Scismic and Overleaf, licensed under [CC BY 4.0](https://creativecommons.org/licenses/by/4.0/), accessed April 10, 2020 from [Overleaf](https://www.overleaf.com/latex/templates/scismics-recommended-cv-template-for-biotech-and-pharma-jobs/hbnkjrjnnpjz)

\documentclass[10pt]{article} % decrease fontsize to `10pt` or `11pt` to make more room


%% Dependencies %%

\usepackage{enumitem}
\usepackage{geometry}
\usepackage{microtype}
\usepackage{hyperref}
\usepackage{paracol}
\usepackage{parskip}
\usepackage{soul}  % \ul
\usepackage{textcase}
\usepackage[compact,small,explicit]{titlesec}  % Used for `titleformat`
\usepackage[x11names]{xcolor}

% Depended upon by `pandoc`'s markdown conversion
\providecommand{\tightlist}{%
  \setlength{\itemsep}{0pt}\setlength{\parskip}{0pt}}


%% Fonts %%

\usepackage[defaultsans]{cantarell}
% Changes the default font family to sans-serif (instead of serif)
\renewcommand*\familydefault{\sfdefault}
% Force using an 8-bit encoding (instead of 7)
\usepackage[T1]{fontenc}


%% Formatting Configuration %%

\setlist{leftmargin=*,nosep}

% Make `\ul` from `soul` look like standard underline
\setul{1pt}{.4pt}

% Turn off ugly blue boxes for links
\hypersetup{colorlinks=true, urlcolor=black}


%% Format Headers %%

% Set custom formatting for H1
\titleformat{\section}%
  {\Large}% Font size
  {\thesection}{0pt}%
  {{%
    \fontfamily{Nunito-TLF}\selectfont% Use rounded font
    \textls[15]{% Increase character spacing
      \MakeTextUppercase{#1}%
    }}%
  }%
  [{\color{gray}\titlerule[0.5pt]}] % Horizontal rule beneath text

% Increase H1 margin-top
\titlespacing{\section}{0pt}{10pt}{5pt}  % left / above / below

% Use rounded font for H2
\titleformat{\subsection}%
  {\large\bfseries\fontfamily{Nunito-TLF}\selectfont}% Font style
  {\thesection}{0pt}%
  {#1}

% Increase H2 margin-top
% \titlespacing{\subsection}{0pt}{14pt}{5pt}  % left / above / below

% Make H3's gray & use rounded font
\titleformat{\subsubsection}
   {\small\bfseries\fontfamily{Nunito-TLF}\selectfont}%
   {\thesection}{0pt}%
   {\textcolor{gray}{#1}}


%% Commands %%

% Store a reference to the original `href` command
% Source: https://tex.stackexchange.com/a/47353
\let\defaultref\href

% Override `href` to add an underline via `\ul`, which fixes underlines without
% needing to use `\smash`--which breaks padding
\renewcommand{\href}[2]{%
  \defaultref{#1}{\ul{#2}}%
}

\newcommand{\link}[2]{\href{#1}{#2}}

% Override `href` to add an underline via `\ul`, which fixes underlines without
% needing to use `\smash`--which breaks padding
\renewcommand{\emph}[1]{%
  \textcolor{gray}{#1}%
}

% Byline with contact info / links
\def\separator{ · {}}

\newcommand{\name}[1]{%
  \begin{center}
      {\huge\bfseries\MakeTextUppercase{#1}\par}
  \end{center}
}

%% Document Configuration %%

\geometry{a4paper,margin=0.5in}
\pagestyle{empty}
\setcounter{secnumdepth}{0}


%% Document metadata %%
% NOTE: Variables come from markdown front-matter via `pandoc` conversion.

\title{Julian Fortune's Resumé}
\author{Julian Fortune}


%% Document definition %%

\begin{document}

% Name (title)
\begin{bfseries}\begin{huge}
  {\fontfamily{Nunito-TLF}\selectfont%
    \MakeTextUppercase{Julian Fortune}
  }
\end{huge}\end{bfseries}

% Contact info
  \link{https://forms.gle/KkSirNbEgQozTH2x7}{Contact}\separator%
% Public info
    \link{http://github.com/julianfortune}{GitHub}\separator%
\link{http://linkedin.com/in/julianfortune}{LinkedIn}%

% \begin{raggedright}

\columnratio{0.75}
\begin{paracol}{2}
\setlength{\columnsep}{1em}

\hypertarget{experience}{%
\section{Experience}\label{experience}}

\hypertarget{amplica-labs}{%
\subsection{Amplica Labs}\label{amplica-labs}}

\vspace{-5pt}

\hypertarget{software-engineer-authentication-team-dec-2023-present}{%
\subsubsection{\texorpdfstring{\small Software Engineer (Authentication
Team) \hfill Dec 2023 --
Present}{Software Engineer (Authentication Team) Dec 2023 -- Present}}\label{software-engineer-authentication-team-dec-2023-present}}

Transferred to a back-end team working on user authentication for a
social media platform (Frequency) based on an open source protocol.

\begin{itemize}
\tightlist
\item
  Added password authentication support to DB layer with salting and
  hashing.
\item
  Implemented phone number blocking based on arbitrary prefixes via a
  Trie.
\item
  Built out a separate service for sending SMS messages with
  fraud-prevention logic.
\item
  \textbf{Stack}: Kotlin, Spring, Redis, \& Postgres
\end{itemize}

\hypertarget{software-engineer-full-stack-august-2023-dec-2023}{%
\subsubsection{\texorpdfstring{\small Software Engineer (Full-stack)
\hfill August 2023 -- Dec
2023}{Software Engineer (Full-stack) August 2023 -- Dec 2023}}\label{software-engineer-full-stack-august-2023-dec-2023}}

Hired to work across the stack and bring functional programming
experience to a customer loyalty product for a French soccer league.

\begin{itemize}
\tightlist
\item
  Developed React components using Tailwind CSS, implemented endpoints,
  added front-end features, and wrote Postgres migrations.
\item
  Mentored peers and championed usage of algebraic data types and
  generic typing.
\item
  Transformed requirements into concrete technical specifications and
  tickets (e.g., auction system), and collaborated with the design and
  product teams.
\item
  \textbf{Stack}: Typescript, React, Storybook, NextJS, and tRPC.
\end{itemize}

\hypertarget{disney-streaming-software-engineer-june-2022august-2023}{%
\subsection{\texorpdfstring{Disney Streaming \emph{· \small Software
Engineer \hfill June 2022--August
2023}}{Disney Streaming · Software Engineer June 2022--August 2023}}\label{disney-streaming-software-engineer-june-2022august-2023}}

Recruited to contribute to a set of internal tools for AWS Kinesis,
including schema registry, code generation, and producer and consumer
SDKs.

\begin{itemize}
\tightlist
\item
  Added Python support to the code generation plugin, which outputs a
  custom library with classes corresponding each event defined in a
  given schema registry.
\item
  Analyzed and optimized AWS usage, resulting in annual savings over
  \$1M.
\item
  Added features to SDK's, `higher-level' services (e.g., snapshotter),
  and SBT plugins.
\item
  Created dashboards that enabled faster resolutions to production
  incidents.
\end{itemize}

\hypertarget{agot-ai}{%
\subsection{Agot AI}\label{agot-ai}}

\vspace{-5pt}

\hypertarget{team-lead-jan-2022-june-2022}{%
\subsubsection{\texorpdfstring{\small Team Lead \hfill Jan 2022 -- June
2022}{Team Lead Jan 2022 -- June 2022}}\label{team-lead-jan-2022-june-2022}}

Promoted to lead the team responsible for processing videos and creating
training labels.

\begin{itemize}
\tightlist
\item
  Planned sprints, conducted interviews, and tracked the team's
  progress.
\item
  Doubled the rate of training data produced through process
  improvements.
\end{itemize}

\hypertarget{software-engineer-june-2021-jan-2022}{%
\subsubsection{\texorpdfstring{\small Software Engineer \hfill June 2021
-- Jan
2022}{Software Engineer June 2021 -- Jan 2022}}\label{software-engineer-june-2021-jan-2022}}

Hired to develop integrations and algorithms for a computer vision
feedback system.

\begin{itemize}
\tightlist
\item
  Implemented microservices in Python communicating via RabbitMQ.
\item
  Wrote Python libraries for standardizing consuming and producing
  RabbitMQ events
\item
  Started a Python lecture series to promote best practices.
\end{itemize}

\hypertarget{skyworks-solutions-software-engineer-intern-jan-2021-june-2021}{%
\subsection{\texorpdfstring{Skyworks Solutions \emph{· \small Software
Engineer (Intern) \hfill Jan 2021 -- June
2021}}{Skyworks Solutions · Software Engineer (Intern) Jan 2021 -- June 2021}}\label{skyworks-solutions-software-engineer-intern-jan-2021-june-2021}}

Brought on to assist the audio deep learning team by performing machine
learning experiments, administering cloud resources (e.g., MongoDB and
TPU instances), and writing tests.

\begin{itemize}
\tightlist
\item
  Created a Python library for defining weighted finite-state
  transducers and applying operations (e.g., composition) based on
  technical papers.
\end{itemize}

\hypertarget{lucid-software-software-engineer-intern-june-2020-sep-2020}{%
\subsection{\texorpdfstring{Lucid Software \emph{· \small Software
Engineer (Intern) \hfill June 2020 -- Sep
2020}}{Lucid Software · Software Engineer (Intern) June 2020 -- Sep 2020}}\label{lucid-software-software-engineer-intern-june-2020-sep-2020}}

Developed new front-end components and refactored endpoints to support
front-end changes.

\begin{itemize}
\tightlist
\item
  \textbf{Stack}: Angular and Typescript frontend, Scala backend.
\end{itemize}

\hypertarget{cbt-nuggets-software-engineer-intern-june-2018-june-2019}{%
\subsection{\texorpdfstring{CBT Nuggets \emph{· \small Software Engineer
(Intern) \hfill June 2018 -- June
2019}}{CBT Nuggets · Software Engineer (Intern) June 2018 -- June 2019}}\label{cbt-nuggets-software-engineer-intern-june-2018-june-2019}}

Worked on the mobile apps team (part-time during college) developing for
iOS and tvOS.

\switchcolumn

\hypertarget{about}{%
\section{About}\label{about}}

Software engineer with 4 years of professional experience, expertise in
leveraging functional techniques and type systems to prevent bugs, and a
track record of success in environments ranging from early startups to
massive enterprises.

\raggedright

\hypertarget{education}{%
\section{Education}\label{education}}

\textbf{Oregon State University}

\hypertarget{b.s.-computer-science-4.0-gpa-20172021}{%
\subsubsection{B.S., Computer Science (4.0 GPA)
2017--2021}\label{b.s.-computer-science-4.0-gpa-20172021}}

Relevant courses: Programming Language Fundamentals (Haskell) \& Deep
Learning.

\hypertarget{projects}{%
\section{Projects}\label{projects}}

\textbf{\href{https://github.com/julianfortune/CS381Project}{Functional-C}}
--- Statically-typed, interpreted language written in Haskell.

\textbf{\href{https://github.com/Tereshchenkolab/paper-ecg}{PaperECG}}
--- An application that digitizes ECGs built with PyQt.

\hypertarget{publications}{%
\section{Publications}\label{publications}}

\vspace{1pt}

\href{https://doi.org/10.1177/1071181320641076}{Real-Time Speech
Workload Estimation for Intelligent Human-Machine Systems}\\
\emph{Human Factors and Ergonomics Society Annual Meeting, 2020.}

\href{https://doi.org/10.1101/2021.07.13.21260461}{Digitizing ECG image:
A new method and open-source software code}\\
\emph{Computer Methods and Programs in Biomedicine, June 2022.}

\hypertarget{skills}{%
\section{Skills}\label{skills}}

\textbf{Languages:} Python, Typescript, Kotlin, Scala (with
\texttt{cats}), Swift, \& Haskell.

\textbf{Cloud:} Object storage, Postgres, MongoDB, Redis, instances, \&
message queues.

\textbf{Tools:} Docker, Terraform, Gradle, \& Make.

\end{paracol}

% \end{raggedright}

\end{document}
