% Adapted from: "Scismic's Recommended CV Template for Biotech and Pharma Jobs" by Scismic and Overleaf, licensed under [CC BY 4.0](https://creativecommons.org/licenses/by/4.0/), accessed April 10, 2020 from [Overleaf](https://www.overleaf.com/latex/templates/scismics-recommended-cv-template-for-biotech-and-pharma-jobs/hbnkjrjnnpjz)

\documentclass[12pt]{article} % decrease fontsize to `10pt` or `11pt` to make more room


%% Dependencies %%

\usepackage{enumitem}
\usepackage{geometry}
\usepackage{microtype}
\usepackage{hyperref}
\usepackage{paracol}
\usepackage{parskip}
\usepackage{soul}  % \ul
\usepackage{textcase}
\usepackage[compact,small,explicit]{titlesec}  % Used for `titleformat`
\usepackage[x11names]{xcolor}

% Depended upon by `pandoc`'s markdown conversion
\providecommand{\tightlist}{%
  \setlength{\itemsep}{0pt}\setlength{\parskip}{0pt}}


%% Fonts %%

\usepackage[defaultsans]{cantarell}
% Changes the default font family to sans-serif (instead of serif)
\renewcommand*\familydefault{\sfdefault}
% Force using an 8-bit encoding (instead of 7)
\usepackage[T1]{fontenc}


%% Formatting Configuration %%

\setlist{leftmargin=*,nosep}

% Make `\ul` from `soul` look like standard underline
\setul{1pt}{.4pt}

% Turn off ugly blue boxes for links
\hypersetup{colorlinks=true, urlcolor=black}


%% Format Headers %%

% Set custom formatting for H1
\titleformat{\section}%
  {\Large}% Font size
  {\thesection}%
  {0pt}%
  {{%
    \fontfamily{Nunito-TLF}\selectfont% Use rounded font
    \textls[15]{% Increase character spacing
      \MakeTextUppercase{#1}%
    }}%
  }%
  [{\color{gray}\titlerule[0.5pt]}] % Horizontal rule beneath text

% Increase H1 margin-top
\titlespacing{\section}{0pt}{10pt}{5pt}  % left / above / below

% Use rounded font for H1
\titleformat{\subsection}%
  {\bfseries\fontfamily{Nunito-TLF}\selectfont}%
  {\thesection}{0pt}{#1}

% Make H3's gray & use rounded font
\titleformat{\subsubsection}
   {\small\bfseries\fontfamily{Nunito-TLF}\selectfont}%
   {\thesection}{0pt}%
   {\textcolor{gray}{#1}}


%% Commands %%

% Store a reference to the original `href` command
% Source: https://tex.stackexchange.com/a/47353
\let\defaultref\href

% Override `href` to add an underline via `\ul`, which fixes underlines without
% needing to use `\smash`--which breaks padding
\renewcommand{\href}[2]{%
  \defaultref{#1}{\ul{#2}}%
}

\newcommand{\link}[2]{\href{#1}{#2}}

% Override `href` to add an underline via `\ul`, which fixes underlines without
% needing to use `\smash`--which breaks padding
\renewcommand{\emph}[1]{%
  \textcolor{gray}{#1}%
}

% Byline with contact info / links
\def\separator{ · {}}

\newcommand{\name}[1]{%
  \begin{center}
      {\huge\bfseries\MakeTextUppercase{#1}\par}
  \end{center}
}

\newcommand{\experience}[3]{%
  \vspace{5pt}%
  \textbf{#1}\separator#2\hfill%
  \textcolor{gray}{#3}%
}


%% Document Configuration %%

\geometry{a4paper,margin=0.5in}
\pagestyle{empty}
\setcounter{secnumdepth}{0}


%% Document metadata %%
% NOTE: Variables come from markdown front-matter via `pandoc` conversion.

\title{Julian Fortune's Resumé}
\author{Julian Fortune}


%% Document definition %%

\begin{document}

% Name (title)
\begin{bfseries}\begin{huge}
  {\fontfamily{Nunito-TLF}\selectfont%
    \MakeTextUppercase{Julian Fortune}
  }
\end{huge}\end{bfseries}

% Contact info
  \link{https://julianfortune.com}{Contact}\separator%
% Public info
    \link{http://github.com/julianfortune}{GitHub}\separator%
\link{http://linkedin.com/in/julianfortune}{LinkedIn}%

\columnratio{0.65}%
\begin{paracol}{2}
\begin{raggedright}

\hypertarget{experience}{%
\section{Experience}\label{experience}}

\hypertarget{disney-streaming-software-engineer-june-2022present}{%
\subsection{\texorpdfstring{Disney Streaming \emph{· \small Software
Engineer \hfill June
2022--Present}}{Disney Streaming · Software Engineer June 2022--Present}}\label{disney-streaming-software-engineer-june-2022present}}

\begin{itemize}
\tightlist
\item
  Member of the Streaming Data Platform team, which provides a suite of
  internal tools for producing and consuming asynchronous events over
  Kinesis such that compatibility is guaranteed.
\item
  Implemented a Python backend (i.e., generates Python code) for the
  team's code generation pipeline.
\item
  Added features to the Scala, Java, \& Python SDK's, as well as the
  team's various integration services.
\item
  Analyzed and optimized AWS infrastructure usage resulting in annual
  cost reductions over \$1M.
\item
  Created dashboards and response plans that enabled faster resolutions
  to production incidents.
\item
  Wrote documentation and created an internal tool to reduce friction
  for platform users.
\end{itemize}

\hypertarget{agot-ai}{%
\subsection{Agot AI}\label{agot-ai}}

\vspace{-5pt}

\hypertarget{team-lead-jan-2022-june-2022}{%
\subsubsection{\texorpdfstring{\small Team Lead \hfill Jan 2022 -- June
2022}{Team Lead Jan 2022 -- June 2022}}\label{team-lead-jan-2022-june-2022}}

\begin{itemize}
\tightlist
\item
  Led the Data team, which produced videos and labels for training
  models.
\item
  Designed schemas and translators for object tracking and
  classification annotations.
\item
  Advocated for strong typing, pure functions, and immutability.
\end{itemize}

\hypertarget{software-engineer-june-2021-jan-2022}{%
\subsubsection{\texorpdfstring{\small Software Engineer \hfill June 2021
-- Jan
2022}{Software Engineer June 2021 -- Jan 2022}}\label{software-engineer-june-2021-jan-2022}}

\begin{itemize}
\tightlist
\item
  Designed and implemented containerized microservices in Python
  communicating via RabbitMQ.
\item
  Reverse-engineered a system consisting of Kotlin-based android app and
  microservices.
\end{itemize}

\hypertarget{skyworks-solutions-intern-jan-2021-june-2021}{%
\subsection{\texorpdfstring{Skyworks Solutions \emph{· \small Intern
\hfill Jan 2021 -- June
2021}}{Skyworks Solutions · Intern Jan 2021 -- June 2021}}\label{skyworks-solutions-intern-jan-2021-june-2021}}

\begin{itemize}
\tightlist
\item
  Performed machine learning experiments, oversaw cloud resources (e.g.,
  MongoDB), \& wrote tests.
\item
  Created a weighted finite-state transducer library that supports
  on-the-fly composition with filters.
\end{itemize}

\hypertarget{lucid-software-intern-june-2020-sep-2020}{%
\subsection{\texorpdfstring{Lucid Software \emph{· \small Intern
\hfill June 2020 -- Sep
2020}}{Lucid Software · Intern June 2020 -- Sep 2020}}\label{lucid-software-intern-june-2020-sep-2020}}

\begin{itemize}
\tightlist
\item
  Redesigned the paywall system using Angular and Typescript to support
  a suite of products.
\item
  Modified endpoints and models in backend (Scala) to support new
  front-end features.
\item
  Participated in a hand-drawn shape detection hackathon project which
  came in 1st place.
\end{itemize}

\switchcolumn 

\hypertarget{skills}{%
\section{Skills}\label{skills}}

\hypertarget{languages}{%
\subsection{Languages}\label{languages}}

\begin{itemize}
\tightlist
\item
  \emph{Expert in} Python
\item
  \emph{Proficient in} Scala, Kotlin, \& Typescript
\item
  \emph{Learning} Haskell, Elm
\end{itemize}

\hypertarget{tech}{%
\subsection{Tech}\label{tech}}

\begin{itemize}
\tightlist
\item
  \emph{Proficient in} Object storage (S3), Instances (EC2), Docker, \&
  Terraform
\item
  \emph{Familiar with} RabbitMQ, Kinesis, \& Kubernetes
\end{itemize}

\hypertarget{education}{%
\section{Education}\label{education}}

\hypertarget{oregon-state-university-2021}{%
\subsection{Oregon State University ·
2021}\label{oregon-state-university-2021}}

\vspace{-5pt}

\hypertarget{b.s.-computer-science-4.0-gpa}{%
\subsubsection{B.S., Computer Science (4.0
GPA)}\label{b.s.-computer-science-4.0-gpa}}

Relevant courses: Programming Language Fundamentals (Haskell) \& Deep
Learning

\hypertarget{projects}{%
\section{Projects}\label{projects}}

\hypertarget{functional-c}{%
\subsection{\texorpdfstring{\href{https://github.com/julianfortune/CS381Project}{Functional-C}}{Functional-C}}\label{functional-c}}

Statically-typed, interpreted, side-effect-free language written in
Haskell.

\hypertarget{paperecg}{%
\subsection{\texorpdfstring{\href{https://github.com/Tereshchenkolab/paper-ecg}{PaperECG}}{PaperECG}}\label{paperecg}}

An application that digitizes electrocardiograms built with OpenCV and
PyQt.

\hypertarget{publications}{%
\section{Publications}\label{publications}}

\vspace{1pt}

Real-Time Speech Workload Estimation for Intelligent Human-Machine
Systems\\
\emph{Human Factors and Ergonomics Society Annual Meeting, 2020.
Co-authored with Dr.~Jamison Heard and Dr.~Julie A. Adams}

Real-time Speech Workload Estimation\\
\emph{Undergraduate Honors Thesis, May 2020}

\end{raggedright}
\end{paracol}

\end{document}
